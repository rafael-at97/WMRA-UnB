\chapter{Conclusões}

\label{CapConclusoes}

Foram demonstrados durante o trabalho uma análise do projeto que serviu como base para este e os projetos
de circuitos de atuação e sensoriamento para o manipulador robótico em questão. As diversas decisões de 
projeto foram orientadas pelo custo de manufatura e montagem dos dispositivos, segurança aos equipamentos e ao 
usuário e pela confiabilidade geral do circuito elétrico e sistema de comando, buscando desenvolver uma 
solução que se mostrasse capaz de ser uma interface entre o usuário e seu ambiente.

A análise do conjunto de atuadores pré-selecionado para o braço robótico e suas respectivas transmissões 
através de engrenagens se mostrou insuficiente para a movimentação correta do sistema. Foi necessário 
propor modificações em algumas juntas para que as necessidades de torque pudessem ser de fato atendidas.

Os circuitos projetados e escolhidos para o sistema demonstraram bons resultados no uso com sensores e atuadores reais. 
A escolha dos equipamentos levou em consideração um equilíbrio entre o custo e integridade 
dos sinais envolvidos, fornecendo dados de leitura corretos mesmo para um protótipo construído com dispositivos e 
condições não ideais, em ambiente de testes sem ferramental próprio para execução no manipulador real.

Em relação ao sistema de comando, foram criados módulos generalizados e intuitivos, para permitir reaproveitamento de código em 
outros projetos. A organização geral e uso de interrupções em \textit{software} permitiu uma resposta rápida do sistema a modificações
das variáveis lidas por sensores e uma simplicidade no fluxograma principal do código, resultando em um aproveitamento
das diversas funcionalidades oferecidas pelo controlador empregado. Esta decisão facilitou ainda a operação
dos atuadores em paralelo, essencial no acionamento de um manipulador robótico.

A técnica de obtenção das velocidades do robô através da matriz jacobiana inversa facilitou a atualização das
referências das velocidades dos motores com base nos dados do usuário. A otimização desta matriz 
em sua forma de função ofereceu resultados satisfatórios, permitindo a operação de inversão desta matriz
relativamente trabalhosa em um ambiente embarcado.

O ambiente de simulação criado inicialmente como um meio de contornar a ausência da estrutura física do
manipulador se mostrou como bom meio de visualizar e validar os equipamentos do mesmo. A comunicação
entre o sistema embarcado e um computador gera uma pequena sobrecarga sobre o Arduino, mas esta é compensada pela capacidade
de utilizar sistema computacional mais potente para realizar estudos avançados sobre a estrutura,
como através das equações de Newton-Euler, já implementadas no módulo de animação, para estimar os torques agindo nos
atuadores.

Todo o material desenvolvido durante este trabalho, inclusive aqueles utilizados na construção deste 
relatório, está agrupado em um repositório \textit{online}, para consultas e confirmação do que foi 
dito e afirmado neste documento. Uma descrição do repositório pode ser vista no apêndice \ref{Anexo-Repositorio}.

\section{Perspectivas Futuras}

% Cinemática inversa
A atuação do manipulador foi comandada completamente pelo uso da geração de velocidades através da matriz
jacobiana inversa. Para facilitar o uso pelo usuário, pensou-se na possibilidade de utilizar um meio de 
salvar posições de maior uso pelo usuário, como posições de alimentação e captura de objetos em determinada 
posição. Tendo em mente esta capacidade, foi realizada a cinemática inversa do manipulador, para direcionar
trabalhos futuros na inclusão desta funcionalidade ao manipulador. A cinemática inversa pode ser vista no apêndice
\ref{Anexo-CinInv}.

% Controle mais avançado
O formato em que o controle dos motores DC foi projetado levou em consideração a possibilidade de uso de técnicas 
de controle mais avançadas. A utilização de uma interrupção temporizada permite o emprego de sistemas de controle
digital com período de atualização bem definido. Para maior confiabilidade na atuação dos motores DC pode ser 
previsto o uso de um controlador adaptativo, resultando em um controlador geral, independente da influência dos 
parâmetros construtivos do motor empregado. Levando em conta a segurança do usuário, poderia ser proposto ainda 
para o sistema um método de implementação de um controle complacente, utilizando dados de torque nas juntas e no
efetuador final a fim de evitar forças desnecessárias na atuação do robô.

% Testes com diferentes efetuadores finais
Em seu estado atual, o manipulador não conta com um efetuador final bem definido, mas sim com um ponto 
de acoplamento. Seria interessante um estudo sobre a possibilidade de aplicação de diversas ferramentas 
pelo sistema. Para interação com esse efetuador, foi deixado no circuito da placa principal um conector
para ser empregado nesta comunicação, fornecendo um sinal com funcionalidade PWM, para uso com servo-motores 
ou qualquer outro atuador que possa se beneficiar deste tipo de sinal.

% Modificações na estrutura
Se mostrando como uma das principais dificuldades encontradas durante o projeto, a estrutura mecânica do 
manipulador poderia ser modificada para aproveitar ao máximo os diversos componentes empregados. A 
modificação das juntas para incluir caixas de redução harmônicas permitiria o uso de motores de baixo 
torque, garantindo uma melhor resposta do sistema, principalmente nas juntas 2, 3 e 4.