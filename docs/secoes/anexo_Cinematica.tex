\chapter{Cinemática Inversa do Manipulador}
\label{Anexo-CinInv}

A cinemática inversa possibilita obter os ângulos a serem atingidos pelas juntas de um
manipulador robótica para que este apresente uma posição e orientação desejadas. É comum
que estes valores desejados sejam informados para o efetuador final do sistema, denominado nestes 
cálculos por $T$, em relação ao sistema de coordenadas da base $B$.
A relação entre estes sistemas de coordenadas pode ser transformada em uma relação 
entre os sistemas de coordenadas da última junta do manipulador em relação à primeira, seguindo a equação 
\ref{eq:tbt}, onde $n$ indica a quantidade de juntas do sistema e $0$ delimita a junta fixa, do sistema.

\begin{equation}
    \label{eq:tbt}
    ^B_TT = ^B_0\!T\;^0_nT\;^n_TT
\end{equation}

As relações $^B_0\!T$ e $^n_TT$ são fixas, representando transformações entre a base fixa do 
robô e uma base lógica do sistema e entre a ferramenta e a última junta, respectivamente. 
Encontrando então a transformação $^0_nT$ a transformação $^B_TT$ é obtida facilmente. 

Para o manipulador deste trabalho, $n=6$, e a transformação de interesse pode ser descrita em 
função de uma matriz de rotação e um vetor de posições, como indica a equação \ref{eq:anexo-t06}.

\begin{equation}
    \label{eq:anexo-t06}
    ^0_6T = 
    \begin{bmatrix}
        r_{11} & r_{12} & r_{13} & p_x \\
        r_{21} & r_{22} & r_{23} & p_y \\
        r_{31} & r_{32} & r_{33} & p_z \\
           0   &    0   &    0   &  1  \\    
    \end{bmatrix}
\end{equation}

Para obter uma relação mais simples da geometria do robô é interessante ter uma relação 
entre a junta 5 a junta 1, portanto, tanto a matriz acima quanto a matriz obtida pela aplicação da
equação \ref{eq:t06} serão multiplicadas por $^0_1T^{-1}$ e $^5_6T^{-1}$ na ordem indicada em \ref{eq:t15}.

\begin{equation}
    \label{eq:t15}
    ^1_5T = ^0_1\!T^{-1}\;^0_6T\;^5_6T^{-1}
\end{equation}

Os passos intermediários para aplicação da equação \ref{eq:t06} estão descritos nas equações de \ref{eq:t46} a
\ref{eq:t06-calc}, cálculos realizados para obter a transformação $^0_6T$ em função dos ângulos internos das juntas.
Multiplicando os dois últimos termos da equação \ref{eq:t06}, obtém-se a transformação descrita na
equação \ref{eq:t46}.

\begin{equation}
    \label{eq:t46}
    ^4_6T = ^4_5T*^5_6T =   \begin{bmatrix}
                                s\theta_5'c\theta_6 & -s\theta_5's\theta_6 & c\theta_5' & c\theta_5'd_6 + a_4 \\
                                s\theta_6 & c\theta_6 & 0 & 0 \\
                                -c\theta_5'c\theta_6 & c\theta_5's\theta_6 & s\theta_5' & s\theta_5'd_6 \\
                                0 & 0 & 0 & 1 \\
                            \end{bmatrix}
\end{equation}

Incluindo a transformação $^3_4T$ no resultado de \ref{eq:t46}, é obtida a equação \ref{eq:t36}.

\begin{equation}
    \label{eq:t36}
    ^3_6T = ^3_4T*^4_6T =   \begin{bmatrix}
                                c\theta_4s\theta_5'c\theta_6 - s\theta_4s\theta_6 & -c\theta_4s\theta_5's\theta_6 - s\theta_4c\theta_6 & c\theta_4c\theta_5' & c\theta_4c\theta_5'd_6 + c\theta_4a_4 + a_3 \\
                                s\theta_4s\theta_5'c\theta_6 + c\theta_4s\theta_6 & -s\theta_4s\theta_5's\theta_6 + c\theta_4c\theta_6 & s\theta_4c\theta_5' & s\theta_4c\theta_5'd_6 + s\theta_4a_4 \\
                                -c\theta_5'c\theta_6 & c\theta_5's\theta_6 & s\theta_5' & s\theta_5'd_6 \\
                                0 & 0 & 0 & 1 \\
                            \end{bmatrix}
\end{equation}

Em \ref{eq:t26} está o resultado obtido após a inclusão de $^2_3T$ na conta.

\begin{equation}
    \label{eq:t26}
    ^2_6T = \begin{bmatrix}
        c(\theta_3+\theta_4)s\theta_5'c\theta_6 & -c(\theta_3+\theta_4)s\theta_5's\theta_6  & c(\theta_3+\theta_4)c\theta_5' & c(\theta_3+\theta_4)(c\theta_5'd_6 + a_4) \\
        -s(\theta_3+\theta_4)s\theta_6          & -s(\theta_3+\theta_4)c\theta_6            &                                & +c\theta_3a_3 + a_2\\
        \\
        s(\theta_3+\theta_4)s\theta_5'c\theta_6 & -s(\theta_3+\theta_4)s\theta_5's\theta_6  & s(\theta_3+\theta_4)c\theta_5' & s(\theta_3+\theta_4)(c\theta_5'd_6 + a_4) \\
        +c(\theta_3+\theta_4)s\theta_6          & +c(\theta_3+\theta_4)c\theta_6            &                                & +s\theta_3a_3 \\
        \\
        -c\theta_5'c\theta_6                    & c\theta_5's\theta_6                       & s\theta_5'                     & s\theta_5'd_6 \\
        \\
        0 & 0 & 0 & 1
    \end{bmatrix}
\end{equation}

Pré-multiplicando a matriz em \ref{eq:t26} por $^1_2T$, chega-se no resultado \ref{eq:t16}.

\begin{equation}
    \label{eq:t16}
    ^1_6T = \begin{bmatrix}
        c(\sum\limits_{i=2}^4\theta_i)s\theta_5'c\theta_6 & -c(\sum\limits_{i=2}^4\theta_i)s\theta_5's\theta_6 & c(\sum\limits_{i=2}^4\theta_i)c\theta_5' & c(\sum\limits_{i=2}^4\theta_i)(c\theta_5'd_6 + a_4) \\
        -s(\sum\limits_{i=2}^4\theta_i)s\theta_6          & -s(\sum\limits_{i=2}^4\theta_i)c\theta_6           &                                         & +c(\theta_2+\theta_3)a_3 + c\theta_2a_2 + a_1\\
        \\
        c\theta_5'c\theta_6                              & -c\theta_5's\theta_6                              & -s\theta_5'                             & -s\theta_5'd_6 \\
        \\
        s(\sum\limits_{i=2}^4\theta_i)s\theta_5'c\theta_6 & -s(\sum\limits_{i=2}^4\theta_i)s\theta_5's\theta_6 & s(\sum\limits_{i=2}^4\theta_i)c\theta_5' & s(\sum\limits_{i=2}^4\theta_i)(c\theta_5'd_6 + a_4) \\
        +c(\sum\limits_{i=2}^4\theta_i)s\theta_6          & +c(\sum\limits_{i=2}^4\theta_i)c\theta_6           &                                         & +s(\theta_2+\theta_3)a_3 + s_2a_2 \\
        \\
        0 & 0 & 0 & 1
    \end{bmatrix}
\end{equation}

Por fim, ao se multiplicar todo o resultado em \ref{eq:t16} pela primeira transformação do sistema, $^0_1T$, é obtida a transformação 
entre a última junta e a base do sistema, disposta em \ref{eq:t06-calc}.

\begin{equation}
    \label{eq:t06-calc}
    ^0_6T = \begin{bmatrix}
        c\theta_1c(\sum\limits_{i=2}^4\theta_i)s\theta_5'c\theta_6 & -c\theta_1c(\sum\limits_{i=2}^4\theta_i)s\theta_5's\theta_6 & c\theta_1c(\sum\limits_{i=2}^4\theta_i)c\theta_5' & c\theta_1[c(\sum\limits_{i=2}^4\theta_i)(c\theta_5'd_6 + a_4) \\
        -c\theta_1s(\sum\limits_{i=2}^4\theta_i)s\theta_6          & -c\theta_1s(\sum\limits_{i=2}^4\theta_i)c\theta_6           & +s\theta_1s\theta_5'                             & +c(\theta_2+\theta_3)a_3 + c\theta_2a_2 + a_1]\\
        -s\theta_1c\theta_5'c\theta_6                             & +s\theta_1c\theta_5's\theta_6                              &                                                  & +s\theta_1s\theta_5'd_6\\
        \\
        s\theta_1c(\sum\limits_{i=2}^4\theta_i)s\theta_5'c\theta_6 & -s\theta_1c(\sum\limits_{i=2}^4\theta_i)s\theta_5's\theta_6 & s\theta_1c(\sum\limits_{i=2}^4\theta_i)c\theta_5' & s\theta_1[c(\sum\limits_{i=2}^4\theta_i)(c\theta_5'd_6 + a_4) \\
        -s\theta_1s(\sum\limits_{i=2}^4\theta_i)s\theta_6          & -s\theta_1s(\sum\limits_{i=2}^4\theta_i)c\theta_6           & -c\theta_1s\theta_5'                             & +c(\theta_2+\theta_3)a_3 + c\theta_2a_2 + a_1]\\
        +c\theta_1c\theta_5'c\theta_6                             & -c\theta_1c\theta_5's\theta_6                              &                                                  & -c\theta_1s\theta_5'd_6\\
        \\
        s(\sum\limits_{i=2}^4\theta_i)s\theta_5'c\theta_6 & -s(\sum\limits_{i=2}^4\theta_i)s\theta_5's\theta_6 & s(\sum\limits_{i=2}^4\theta_i)c\theta_5' & s(\sum\limits_{i=2}^4\theta_i)(c\theta_5'd_6 + a_4) \\
        +c(\sum\limits_{i=2}^4\theta_i)s\theta_6          & +c(\sum\limits_{i=2}^4\theta_i)c\theta_6           &                                         & +s(\theta_2+\theta_3)a_3 + s\theta_2a_2 \\
        \\
        0 & 0 & 0 & 1
    \end{bmatrix}
\end{equation}

Realizando a transformação indicada em \ref{eq:t15} para a matriz da equação \ref{eq:t06-calc}, é obtido o resultado
disposto em \ref{eq:t15-calc}.

\begin{equation}
    \label{eq:t15-calc}
    ^1_5T = 
    \begin{bmatrix}
        s\theta_5'c(\sum\limits_{i=2}^4\theta_i) & c\theta_5'c(\sum\limits_{i=2}^4\theta_i) & s(\sum\limits_{i=2}^4\theta_i)  & c(\sum\limits_{i=2}^4\theta_i)a_4 + c(\theta_2+\theta_3)a_3 + c\theta_2a_2 + a_1 \\
        c\theta_5'          & -s\theta_5'         & 0          & 0 \\
        s\theta_5's(\sum\limits_{i=2}^4\theta_i) & c\theta_5's(\sum\limits_{i=2}^4\theta_i) & -c(\sum\limits_{i=2}^4\theta_i) & s(\sum\limits_{i=2}^4\theta_i)a_4 + s(\theta_2+\theta_3)a_3 + s\theta_2a_2 \\
           0   &    0   &    0   &  1  \\    
    \end{bmatrix}
\end{equation}

Agora, para a matriz $^0_6T$ com base nos parâmetros desejados, seguindo a transformação em \ref{eq:anexo-t06}, 
foi obtido o resultado expresso na equação \ref{eq:t15-calc2}.

\begin{equation}
    \label{eq:t15-calc2}
    ^1_5T = 
    \begin{bmatrix}
         c\theta_1(r_{11}c\theta_6-r_{12}s\theta_6) &  c\theta_1r_{13}+s\theta_1r_{23} & -c\theta_1(r_{11}s\theta_6+r_{12}c\theta_6) &  c\theta_1(p_x-r_{13}d_6) \\
        +s\theta_1(r_{21}c\theta_6-r_{22}s\theta_6) &                      & -s\theta_1(r_{21}s\theta_6+r_{22}c\theta_6) & +s\theta_1(p_y-r_{23}d_6) \\
        & & & & \\
        -s\theta_1(r_{11}c\theta_6-r_{12}s\theta_6) & -s\theta_1r_{13}+c\theta_1r_{23} &  s\theta_1(r_{11}s\theta_6+r_{12}c\theta_6) & -s\theta_1(p_x-r_{13}d_6) \\
        +c\theta_1(r_{21}c\theta_6-r_{22}s\theta_6) &                      & -c\theta_1(r_{21}s\theta_6+r_{22}c\theta_6) & +c\theta_1(p_y-r_{23}d_6) \\
        & & & & \\        
             r_{31}c\theta_6-r_{32}s\theta_6  &         r_{33}       &     -r_{31}s\theta_6-r_{32}c\theta_6  &      p_z-r_{33}d_6  \\
        & & & & \\
                    0             &            0         &              0            &           1         \\    
    \end{bmatrix}
\end{equation}

Os termos $p_x-r_{13}d_6$ e $p_y-r_{23}d_6$ da equação \ref{eq:t15-calc2} equivalem, respectivamente, 
às posições $x$ e $y$ da quinta junta no sistema de coordenadas da primeira junta, assim, alternando esses 
termos para suas respectivas coordenadas polares, obtém-se as relações dispostas em \ref{eq:polares}. Nestas
equações, o termo $\rho$ se refere ao módulo do vetor conectando a base das juntas 1 e 5 projetado sobre o 
plano horizontal $XY$, e $\phi$ o ângulo desta projeção no mesmo plano.

\begin{align}
    \label{eq:polares}
    p_x - r_{13}d_6 &= \; ^1_5\rho_x = \; ^1_5\rho \cdot cos ^1_5\phi \nonumber\\
    p_y - r_{23}d_6 &= \; ^1_5\rho_y = \; ^1_5\rho \cdot sen ^1_5\phi \nonumber\\
\end{align}

Substituindo os valores mais a direita das equações em \ref{eq:polares} no termo (2, 4) 
da matriz em \ref{eq:t15-calc2}, e igualando o resultado ao termo (2, 4) da matriz em 
\ref{eq:t15-calc}, chega-se à relação escrita em \ref{eq:anexo-angulo1}.

\begin{equation}
    \label{eq:anexo-angulo1}
    -sen\theta_1 \cdot ^1_5\rho \cdot cos^1_5\phi + cos\theta_1 \cdot ^1_5\rho \cdot sen^1_5\phi = 0
\end{equation}

Simplificando o resultado em \ref{eq:anexo-angulo1} por vias trigonométricas, chega-se ao resultado em 
\ref{eq:anexo-angulo1-2}.

\begin{equation}
    \label{eq:anexo-angulo1-2}
    ^1_5\rho \cdot sen(\theta_1 - ^1_5\phi) = 0
\end{equation}

Assumindo que o módulo $^1_5\rho$ será sempre positivo, e utilizando a identidade trigonométrica fundamental, 
obtém-se também a igualdade em \ref{eq:anexo-angulo1-3}.

\begin{equation}
    \label{eq:anexo-angulo1-3}
    cos(\theta_1 - ^1_5\phi) = \pm 1
\end{equation}

Essas duas equações, \ref{eq:anexo-angulo1-2} e \ref{eq:anexo-angulo1-3}, levam ao resultado em 
\ref{eq:anexo-angulo1-final}, que indica o valor de $\theta_1$ para a posição desejada. Na equação
é utilizada a função $atan2$, que utiliza o sinal das componentes para estimar corretamente o 
ângulo nos 4 quadrantes do círculo unitário.

\begin{equation}
    \label{eq:anexo-angulo1-final}
    \theta_1 = ^1_5\phi + atan2\left(0, \pm1\right)
\end{equation}

Para a determinação do ângulo $\theta_5'$, basta utilizar o termo (2, 2) e a soma dos quadrados 
dos termos (1, 2) e (3, 2) para as matrizes em \ref{eq:t15-calc} e \ref{eq:t15-calc2}. 
Ao igualar o termo (2, 2), obtém-se a igualdade em \ref{eq:anexo-angulo5}, já ao igualar 
a soma dos quadrados citada, é obtida a equação \ref{eq:anexo-angulo5-2}. 

\begin{equation}
    \label{eq:anexo-angulo5}
    s\theta_5' = s\theta_1r_{13} - c\theta_1r_{23}
\end{equation}

\begin{equation}
    \label{eq:anexo-angulo5-2}
    c\theta_5' = \pm \sqrt{(c\theta_1r_{13} + s\theta_1r_{23})^2 + r_{33}^2}
\end{equation}

Com o valor de $\theta_1$ conhecido, possíveis valores de $\theta_5'$ podem ser obtidos 
através de uma função arco tangente que utilize os termos encontrados em \ref{eq:anexo-angulo5}
e \ref{eq:anexo-angulo5-2}.

Para encontrar o ângulo $\theta_6$, é realizada uma multiplicação entre a matriz em 
\ref{eq:t15-calc2} e $^5_6T$, obtendo a matriz em \ref{eq:t16-calc2}.

\begin{equation}
    \label{eq:t16-calc2}
    ^1_6T = 
    \begin{bmatrix}
         c\theta_1r_{11} + s\theta_1r_{21} &  c\theta_1r_{12} + s\theta_1r_{22} & c\theta_1r_{13} + s\theta_1r_{23} & c\theta_1p_x + s\theta_1p_y \\
        -s\theta_1r_{11} + c\theta_1r_{21} & -s\theta_1r_{12} + c\theta_1r_{22} & -s\theta_1r_{13} +c\theta_1r_{23} & -s\theta_1p_x + c\theta_1p_y \\
        r_{31}                             &         r_{32}                     &     r_{33}                        &      p_z  \\
         0             &            0         &              0            &           1         \\    
    \end{bmatrix}
\end{equation}

Comparando os termos (2, 1) e (2, 2) de \ref{eq:t16-calc2} com os mesmos termos em \ref{eq:t16}, são obtidos valores para $\theta_6$
em função de $\theta_1$ e $\theta_5'$, assim como demonstram as igualdades em \ref{eq:anexo-angulo6}.

\begin{align}
    \label{eq:anexo-angulo6}
    c\theta_6 &= \frac{-s\theta_1r_{11} + c\theta_1r_{21}}{c\theta_5'} \nonumber\\
    s\theta_6 &= \frac{s\theta_1r_{12} - c\theta_1r_{22}}{c\theta_5'} \nonumber\\
\end{align}

As equações em \ref{eq:anexo-angulo6} podem ser reagrupadas de maneira a retirar a dependência no 
ângulo $\theta_5$, assim como mostrado na equação \ref{eq:anexo-angulo6-2}.

\begin{equation}
    \label{eq:anexo-angulo6-2}
    \theta_6 = tan^{-1}\frac{(-s\theta_1r_{11} + c\theta_1r_{21})}{(s\theta_1r_{12} - c\theta_1r_{22})}
\end{equation}

Os ângulos de 2 a 4 foram obtidos com cálculos semelhantes aos que são realizados para a cinemática 
inversa de um manipulador coplanar com 3 eixos. O valor da soma desses ângulos é obtido pela 
comparação dos termos (3, 3) e (1, 3) das matrizes \ref{eq:t15-calc} e \ref{eq:t15-calc2}. 
Essa comparação pode ser vista no conjunto de equações \ref{eq:anexo-angulo234}. Utilizando
a função arcotangente é então possível obter um valor númerico para $\sum\limits_{i=2}^4\theta_i$.

\begin{align}
    \label{eq:anexo-angulo234}
    c(\sum\limits_{i=2}^4\theta_i) &= s\theta_6r_{31} + c\theta_6r_{32} \nonumber\\
    s(\sum\limits_{i=2}^4\theta_i) &= -c\theta_1(s\theta_6r_{11}+c\theta_6r_{12}) - s\theta_1(s\theta_6r_{21}+c\theta_6r_{22}) \nonumber\\
\end{align}

Comparando agora os termos (1, 4) e (3, 4) das mesmas matrizes, são obtidas as igualdades dispostas em 
\ref{eq:anexo-angulo3}.

\begin{align}
    \label{eq:anexo-angulo3}
    c(\sum\limits_{i=2}^4\theta_i)a_4 + c(\theta_2+\theta_3)a_3 + c\theta_2a_2& + a_1 = c\theta_1(p_x-r_{13}d_6) + s\theta_1(p_y-r_{23}d_6) \nonumber\\
    s(\sum\limits_{i=2}^4\theta_i)a_4 + s(\theta_2+\theta_3)a_3 + s\theta_2a_2&       = p_z - r_{33}d_6 \nonumber\\
\end{align}

Projetando o vetor que liga a base das juntas 1 e 5 em um plano vertical, que contém os eixos dos 
elos que conectam as juntas 2, 3 e 4, nota-se que é possível definir duas coordenadas, $\mathcal{X}$
e $\mathcal{Y}$, que se relacionam com o conjunto de equações em \ref{eq:anexo-angulo3} do modo
descrito em \ref{eq:anexo-angulo3-2}.

\begin{align}
    \label{eq:anexo-angulo3-2}
    \mathcal{X} &= c\theta_1(p_x-r_{13}d_6) + s\theta_1(p_y-r_{23}d_6) - c(\sum\limits_{i=2}^4\theta_i)a_4 -a_1 = c(\theta_2+\theta_3)a_3 + c\theta_2a_2\nonumber\\
    \mathcal{Y} &= p_z - r_{33}d_6 - s(\sum\limits_{i=2}^4\theta_i)a_4 = s(\theta_2+\theta_3)a_3 + s\theta_2a_2 \nonumber\\
\end{align}

Elevando ambos termos $\mathcal{X}$ e $\mathcal{Y}$ ao quadrado e realizando a soma dos resultados, 
é possível obter um valor para o cosseno de $\theta_3$, assim como descrito em \ref{eq:anexo-angulo3-3}.
O valor do ângulo em si pode ser obtido por uma função trigonométrica inversa.

\begin{equation}
    \label{eq:anexo-angulo3-3}
    c\theta_3 = \frac{\mathcal{X}^2 + \mathcal{Y}^2 - a_2^2 - a_3^2}{2a_2a_3}
\end{equation}

Sabendo o valor de $\theta_3$ é possível reescrever as equações de $\mathcal{X}$ e $\mathcal{Y}$ utilizando 
duas constantes auxiliares, $k_1$ e $k_2$, onde $k_1=c\theta_3a_3 + a_2$ e $k_2=s\theta_3a_3$. Assim, as 
variáveis são expressas no novo modo, disposto em \ref{eq:anexo-angulo2}.

\begin{align}
    \label{eq:anexo-angulo2}
    \mathcal{X} &= k_1c\theta_2 - k_2s\theta_2 \nonumber\\
    \mathcal{Y} &= k_1s\theta_2 + k_2c\theta_2 \nonumber\\
\end{align}

Utilizando mais outras duas variáveis auxiliares, $\mathcal{T}=\sqrt{k_1^2+k_2^2}$ e $\mathcal{K}=atan2(k_2, k_1)$,
de tal modo que $k_1=\mathcal{T}\cdot cos(\mathcal{K})$ e $k_2=\mathcal{T}\cdot sen(\mathcal{K})$, é 
possível escrever equações mais simples para o ângulo $\theta_2$. Estas novas equações estão dispostas em 
\ref{eq:anexo-angulo2-2}.

\begin{align}
    \label{eq:anexo-angulo2-2}
    \mathcal{X} &= \mathcal{T} \cdot cos(\theta_2+\mathcal{K}) \nonumber\\
    \mathcal{Y} &= \mathcal{T} \cdot sen(\theta_2+\mathcal{K}) \nonumber\\
\end{align}

$\theta_2$ é por fim definido pela equação \ref{eq:anexo-angulo2-3}, resultado de análise das equações em 
\ref{eq:anexo-angulo2-2}.

\begin{equation}
    \label{eq:anexo-angulo2-3}
    \theta_2 = atan2(\mathcal{Y}, \mathcal{X}) - \mathcal{K} = atan2(\mathcal{Y}, \mathcal{X}) - atan2(k_2, k_1)
\end{equation}

Tendo em mãos o valor da soma de $\theta_2, \theta_3$ e $\theta_4$, bem como os valores individuais de $\theta_2$ 
e $\theta_3$, o valor de $\theta_4$ é facilmente obtido através de uma simples subtração.